
\documentclass[11pt]{article}
\usepackage{graphicx}

\usepackage{times}
\usepackage{listings}
\usepackage{parcolumns}

\setlength{\parindent}{0in}
\setlength{\parskip}{0.1in}

\title{R Design Patterns: Base-R vs.\ Tidyverse}
\author{Norman Matloff \\
      Dept. of Computer Science, University of California, Davis}

\begin{document}

\maketitle

This document enables the reader to see at a glance the difference
between base-R and the Tidyverse in common R design settings.

All examples use R's built-in datasets.  After e.g., changing a data
frame, it is restored for the next example, e.g. \textbf{data(mtcars)}.

As this document is aimed at comparing base-R and the tidyverse in terms
of teaching new R learners, advanced functions from either base-R or the
tidyverse are excluded here.

More and more examples will be added over time.

\section*{Reading a specific cell in a data frame}

% use by invoking "source ParColsEnv.tex" from Vim


\begin{parcolumns}[rulebetween=true]{2}

\colchunk{

\begin{minipage}{0.40\linewidth}

\begin{lstlisting}

mtcars$mpg[3]

\end{lstlisting}

\end{minipage}

}

\hspace{0.1in}

\colchunk{

\begin{minipage}{0.40\linewidth}

\begin{lstlisting}

mtcars %>% select(mpg) %>% 
  filter(row_number() == 3)

\end{lstlisting}

\end{minipage}

}

\end{parcolumns}

\section*{Adding a column to a data frame}

\begin{parcolumns}[rulebetween=true]{2}

\colchunk{

\begin{minipage}{0.40\linewidth}

\begin{lstlisting}

mtcars$hwratio 
  <- mtcars$hp / mtcars$wt

\end{lstlisting}

\end{minipage}

}

\hspace{0.1in}

\colchunk{

\begin{minipage}{0.40\linewidth}

\begin{lstlisting}

mtcars 
  %>% mutate(hwratio=hp/wt) -> mtcars

\end{lstlisting}

\end{minipage}

}

\end{parcolumns}

\section*{Deleting a column from a data frame}

% use by invoking "source ParColsEnv.tex" from Vim

\begin{parcolumns}[rulebetween=true]{2}

\colchunk{

\begin{minipage}{0.40\linewidth}

\begin{lstlisting}

mtcars$carb <- NULL

\end{lstlisting}

\end{minipage}

}

\hspace{0.1in}

\colchunk{

\begin{minipage}{0.40\linewidth}

\begin{lstlisting}

mtcars %>% select(-carb) -> mtcars

\end{lstlisting}

\end{minipage}

}

\end{parcolumns}

\section*{Deleting multiple columns from a data frame}

\begin{parcolumns}[rulebetween=true]{2}

\colchunk{

\begin{minipage}{0.40\linewidth}

\begin{lstlisting}

mtcars[c('drat','carb')] 
  <- NULL

\end{lstlisting}

\end{minipage}

}

\hspace{0.1in}

\colchunk{

\begin{minipage}{0.40\linewidth}

\begin{lstlisting}

mtcars %>% select(-c(drat,carb)) 
  -> mtcars

\end{lstlisting}

\end{minipage}

}

\end{parcolumns}


\section*{Binary categorization on a vector}

\begin{parcolumns}[rulebetween=true]{2}

\colchunk{

\begin{minipage}{0.40\linewidth}

\begin{lstlisting}
NileHiLow <- 
  ifelse(Nile >= 1000,
    'high','low')
\end{lstlisting}

\end{minipage}

}

\hspace{0.1in}

\colchunk{

\begin{minipage}{0.40\linewidth}

\begin{lstlisting}

Nile %>% as.data.frame %>% 
  mutate(
    HighLow = case_when
    (x < 1000~'low',
     x >= 1000~'high')
  ) %>%
  select(HighLow) %>%
  as.vector -> HighLow


\end{lstlisting}

\end{minipage}

}

\end{parcolumns}

The step of conversion back to a vector at the end is needed for many R
packages in which vector input is required.  

\end{document}

